\documentclass[oneside]{thesisclass}
% Based on thesisclass.cls of Timo Rohrberg, 2009
% ----------------------------------------------------------------
% Thesis - Main document
% ----------------------------------------------------------------




%% -----------------------------
%% | Information for titlepage |
%% -----------------------------

\newcommand{\myname}{Jan Andre Rudolph}
\newcommand{\mytitle}{Matrices \& Mazes \\in Quantum Algorithms}
\newcommand{\semesterD}{Sommersemester 2017}
\newcommand{\semesterE}{Summer term 2017}


%%%%%%%%%%%%%%%%%%%%%%%%%%%%%%%%%
%% Here, main documents begins %%
%%%%%%%%%%%%%%%%%%%%%%%%%%%%%%%%%
\begin{document}

\selectlanguage{english}

\frontmatter
\pagenumbering{roman}
%% titlepage.tex
%%

% coordinates for the bg shape on the titlepage
\newcommand{\diameter}{20}
\newcommand{\xone}{-15}
\newcommand{\xtwo}{160}
\newcommand{\yone}{15}
\newcommand{\ytwo}{-253}

\begin{titlepage}
% bg shape
\begin{tikzpicture}[overlay]
\draw[color=gray]  
 		 (\xone mm, \yone mm)
  -- (\xtwo mm, \yone mm)
 arc (90:0:\diameter pt) 
  -- (\xtwo mm + \diameter pt , \ytwo mm) 
	-- (\xone mm + \diameter pt , \ytwo mm)
 arc (270:180:\diameter pt)
	-- (\xone mm, \yone mm);
\end{tikzpicture}
	\begin{textblock}{10}[0,0](4,2.5)
		\includegraphics[width=.3\textwidth]{logos/KITLogo_RGB.pdf}
	\end{textblock}
	\changefont{phv}{m}{n}	% helvetica	
	\vspace*{3.5cm}
	\begin{center}
		\Huge{\mytitle}
		\vspace*{2cm}\\
		\Large{
			\iflanguage{english}{by} {von}
		}\\
		\vspace*{1cm}
		\huge{\myname}\\
		\vspace*{4cm}
		\Large{Seminar}\\
                \vspace{1cm}
		\huge{Quantum Algorithms}\\
		\Large{
                \vspace{1cm}
                \Large{Institute of Theoretical Informatics, KIT}\\
			\iflanguage{english}{\semesterE} {\semesterD}
                \vspace{1cm}
		}
	\end{center}
	\vspace*{1cm}

\vspace{2cm}

\begin{textblock}{10}[0,0](4,16.8)
\tiny{ 
	\iflanguage{english}
		{KIT -- University of the State of Baden-Wuerttemberg and National Research Center of the Helmholtz Association}
		{KIT -- Universit\"at des Landes Baden-W\"urttemberg und nationales Forschungszentrum in der Helmholtz-Gemeinschaft}
}
\end{textblock}

\begin{textblock}{10}[0,0](14,16.75)
\large{
	\textbf{www.kit.edu} 
}
\end{textblock}

\end{titlepage}



%% -------------------
%% |   Directories   |
%% -------------------
\tableofcontents


%% -----------------
%% |   Main part   |
%% -----------------
\mainmatter
\pagenumbering{arabic}
\chapter{Introduction}
\nocite{*}
The operationality of Quantum Computers will progress quickly in the next years. 
When Moores Law reaches the "firewall" of atomar physics, the quantum physics will lead us to a new kind of information, carried by the so-called q-bits. 
Therefore the importance of understanding quantum algorithms rises. 
All computer scientists should at least be aware of the possibilities quantum algorithms provide, because the work of today may be affected by quantum computing in the near future. \\
\\There is not yet a high level quantum coding language that compiles in some sense, but we have three basic options to represent quantum operations.
There are so called quantum circuits that just represent the physical structure. 
But unlike electronic circuits, these are very complicated to interpret.
Due to the quantum entanglement, one cannot liniarily read the circuit. 
Merely one has to understand the whole circuit at once, including all measurements. 
If we do not want to understand the complete circuit before we understand one part of it, we can use the linear algebra. \\
\\We will see that we can use matrices to represent circuits or smaller modules of them. 
Not only that mathematics is a global language with the advantages of provability and that computers understand it.
But we will also see that we now have the possibility to explicitly represent, calculate and understand quantum entanglement.
The only remaining problem (besides scalability) is that matrices do not provide a natural image nor the idea of what happens.\\
\\This is where the mazes as our third option of representation have their might.
The so called mazes are a graph based translation of the matrices just like adjacency matrices and the corresponding adjacency graphs - just a little more complicated.
We will introduce some mice as a guide throu the mazes and as an anchor for the imagination.\\
\\This Thesis aims to briefly explain the concepts of matrices and mazes and connects them to the idea of circuits. 
Some important matrices for quantum algorithms will be introduced and explained - also by the use of mazes.
The text is mainly inspired by the Book from R.J.Lipton and K.W.Regan.
It could target any kind of computer scientists or persons with an equal level of mathematics and technology.
\chapter{Linear Algebra}

%\section{Hilbert Space}
Before we can talk about matrices, we need to consider the role of vectors and matrices in the model.
To understand the Hilbert space we are moving in, it seems a good first step to have a look at the Euclidean space.
The Euclidean space is in general n-dimensional.
Normally one describes geometrical objects as subsets of the space, where vectors correspond to points.
With the inner product, the Euclidean space provides the possibility to measure angles and distances.
To represent quantum states, we need complex numbers and therefore the Hilbert space.
The scalar product and the transponition now need the complex conjugated of the imaginary numbers but in general the operations are the same.\\
\\To represent an algorithm we now use a vector as input.
We multiply one or more of oir matrices on the left side and the result is again a vector.
This vector represents the output.

\section{Feasibility}
But we cannot build just any matrix in an actual quantum circuit.
Or in other words not all matrices are quantum feasible.
To determine which matrices are feasible, we first look at the definition of feasibility for quantum circuits.
%TODO continue

\chapter{Boolean Functions}
%TODO mention that mazes are simple
The first class of matrices are those representing boolean functions.
Well, actually those matrices do not represent the functions themselfs, but we will need some boundary conditions to succeed.
These conditions must predefine how to interpret the input aud output vector.
But not all boolean functions are quantum feasible and for some it is not yet clear if they are.
So in the following there will be some matrices introduced and then the link to boolean functions is created.

\section{SWAP}
The good thing about matrices of this class is that they are only filled with ones and zeros.
So we have a kind of easy introduction.
And what is the easiest matrix one can imagine?
Well the identity - this means that the input equals the output.
If we now swap the columns in the middle as shown in figure %TODO grafic
we have to look at the binary strings to understand what physically happens.
For no input or input on both lanes, the behaviour equals the identity.
But if there is only input on one lane, the '1' in the SWAP matrix connects the binary strings to their inverse, respectively '01' -> '10' and '10' -> '01'.
And this is just what the corresponding maze%TODO grafic
shows us - the edges connect%TODO complete
So in general this matrix swaps the two lanes.

\section{CNOT}
%TODO explain CNOT == controled not
Due to the superpositioning, the behaviour heavily changes if we choose to swap two other columns as shown in figure%TODO grafic
as matrix - the CNOT matrix - and in figure%TODO grafic
as a maze.
As we can easity read in the maze, the second lane - the LSB in the boolean string - is toggled when and only when the first lane - the MSB in the boolean string - is one.\\
\\As already mentioned, we can use this behaviour to implement some boolean functions.
Let us do this for the XOR function as an example.\\
\\Since we cannot change the behaviour of the CNOT gate, we have to find a convention for the in- and output that fits the desired function.
We need both lanes as input and one of the two output lanes that is interpreted as the result.
Refering to the maze again, we can just try out both lanes.
If we choose the first lane, the output for XOR(0,1) would be 0 since '01' is connected to '01' where 0 is on the first lane.
This obviously does not match the XOR function.
But for the second lane regrded as output, we obtain the results we expected.\\
\\Another example is the NOT function - which only takes one input lane.
Looking at the maze, we directly see that the second lane is toggled when the first lane is '1'.
This is what we want to achieve - so we set the first lane '1' and regard the second lane as in- and output.
This way the input is always toggled.

\section{Toffoli Gate}
If we extend the CNOT gate by another control lane we call it CCNOT - controled controled not - also known as Toffoli gate.
This gate has three lanes.
Therefore the corresponding matrix - as well as the maze - displays 2^3 superpositions.
So it is of order 8.
Looking at the matrix%TODO grafic
it might not be intuitive where the 'controled controled' comes from.
But again, the maze%TODO grafic
is more clear. 
All nodes are connected to their identity, only the last two are toggled.
In other words - if you look at the boolean strings of these nodes - the input of the last physical lane is toggled when and only when the first two lanes are one.

\section{Controled M}

\chapter{Hadamard Matrix}

\chapter{Fourier Matrix}

\chapter{Conclusions}

%% --------------------
%% |   Bibliography   |
%% --------------------

\bibliographystyle{plain}
\bibliography{mybib.bib}

\end{document}
