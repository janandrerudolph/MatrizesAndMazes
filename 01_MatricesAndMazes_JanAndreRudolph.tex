\documentclass[oneside]{thesisclass}
% Based on thesisclass.cls of Timo Rohrberg, 2009
% ----------------------------------------------------------------
% Thesis - Main document
% ----------------------------------------------------------------




%% -----------------------------
%% | Information for titlepage |
%% -----------------------------

\newcommand{\myname}{Jan Andre Rudolph}
\newcommand{\mytitle}{Matrices \& Mazes \\in Quantum Algorithms}
\newcommand{\semesterD}{Sommersemester 2017}
\newcommand{\semesterE}{Summer term 2017}


%%%%%%%%%%%%%%%%%%%%%%%%%%%%%%%%%
%% Here, main documents begins %%
%%%%%%%%%%%%%%%%%%%%%%%%%%%%%%%%%
\begin{document}

\selectlanguage{english}

\frontmatter
\pagenumbering{roman}
%% titlepage.tex
%%

% coordinates for the bg shape on the titlepage
\newcommand{\diameter}{20}
\newcommand{\xone}{-15}
\newcommand{\xtwo}{160}
\newcommand{\yone}{15}
\newcommand{\ytwo}{-253}

\begin{titlepage}
% bg shape
\begin{tikzpicture}[overlay]
\draw[color=gray]  
 		 (\xone mm, \yone mm)
  -- (\xtwo mm, \yone mm)
 arc (90:0:\diameter pt) 
  -- (\xtwo mm + \diameter pt , \ytwo mm) 
	-- (\xone mm + \diameter pt , \ytwo mm)
 arc (270:180:\diameter pt)
	-- (\xone mm, \yone mm);
\end{tikzpicture}
	\begin{textblock}{10}[0,0](4,2.5)
		\includegraphics[width=.3\textwidth]{logos/KITLogo_RGB.pdf}
	\end{textblock}
	\changefont{phv}{m}{n}	% helvetica	
	\vspace*{3.5cm}
	\begin{center}
		\Huge{\mytitle}
		\vspace*{2cm}\\
		\Large{
			\iflanguage{english}{by} {von}
		}\\
		\vspace*{1cm}
		\huge{\myname}\\
		\vspace*{4cm}
		\Large{Seminar}\\
                \vspace{1cm}
		\huge{Quantum Algorithms}\\
		\Large{
                \vspace{1cm}
                \Large{Institute of Theoretical Informatics, KIT}\\
			\iflanguage{english}{\semesterE} {\semesterD}
                \vspace{1cm}
		}
	\end{center}
	\vspace*{1cm}

\vspace{2cm}

\begin{textblock}{10}[0,0](4,16.8)
\tiny{ 
	\iflanguage{english}
		{KIT -- University of the State of Baden-Wuerttemberg and National Research Center of the Helmholtz Association}
		{KIT -- Universit\"at des Landes Baden-W\"urttemberg und nationales Forschungszentrum in der Helmholtz-Gemeinschaft}
}
\end{textblock}

\begin{textblock}{10}[0,0](14,16.75)
\large{
	\textbf{www.kit.edu} 
}
\end{textblock}

\end{titlepage}



%% -------------------
%% |   Directories   |
%% -------------------
\tableofcontents


%% -----------------
%% |   Main part   |
%% -----------------
\mainmatter
\pagenumbering{arabic}
\chapter{Introduction}
\nocite{*}
The operationality of Quantum Computers will progress quickly in the next years. 
When Moores Law reaches the "firewall" of atomar physics, the quantum physics will lead us to a new kind of information, carried by the so-called q-bits. 
Therefore the importance of understanding quantum algorithms rises. 
All computer scientists should at least be aware of the possibilities quantum algorithms provide, because the work of today may be affected by quantum computing in the near future. \\
\\There is not yet a high level quantum coding language that compiles in some sense, but we have three basic options to represent quantum operations.
There are so called quantum circuits that just represent the physical structure. 
But unlike electronic circuits, these are very complicated to interpret.
Due to the quantum entanglement, one cannot liniarily read the circuit. 
Merely one has to understand the whole circuit at once, including all measurements. 
If we do not want to understand the complete circuit before we understand one part of it, we can use the linear algebra. \\
\\We will see that we can use matrices to represent circuits or smaller modules of them. 
Not only that mathematics is a global language with the advantages of provability and that computers understand it.
But we will also see that we now have the possibility to explicitly represent, calculate and understand quantum entanglement.
The only remaining problem (besides scalability) is that matrices do not provide a natural image nor the idea of what happens.\\
\\This is where the mazes as our third option of representation have their might.
The so called mazes are a graph based translation of the matrices just like adjacency matrices and the corresponding adjacency graphs - just a little more complicated.
We will introduce some mice as a guide throu the mazes and as an anchor for the imagination.\\
\\This Thesis aims to briefly explain the concepts of matrices and mazes and connects them to the idea of circuits. 
Some important matrices for quantum algorithms will be introduced and explained - also by the use of mazes.
The text is mainly inspired by the Book from R.J.Lipton and K.W.Regan.
It could target any kind of computer scientists or persons with an equal level of mathematics and technology.
\chapter{Linear Algebra}

\chapter{Boolean Functions}

\section{CNOT \& SWAP}

\section{Toffoli Gate}

\chapter{Hadamard Matrix}

\chapter{Fourier Matrix}

\chapter{Conclusions}

%% --------------------
%% |   Bibliography   |
%% --------------------

\bibliographystyle{plain}
\bibliography{mybib.bib}

\end{document}